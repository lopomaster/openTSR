% Este archivo es parte de la memoria del proyecto fin de carrera
% de Diego Barrios Romero. Protegida bajo la licencia GFDL.
% Para más información, la licencia completa viene incluida en el
% fichero fdl-1.3.tex

% Copyright (C) 2009 Diego Barrios Romero

\chapter{Pruebas}
\label{chap:pruebas}
La metodología de desarrollo de este juego ha sido basada en
componentes por lo que los componentes se han ido desarrollando
haciendo pequeñas pruebas sobre ellos a medida que se iban
implementando nuevas funcionalidades. Por tanto, no existe la
tradicional fase de pruebas propiamente dicha sino que éstas se han
ido haciendo poco a poco a lo largo del desarrollo y no existen como
una fase separada.

Durante el desarrollo de este juego no he podido probarlo sobre una
nintendo DS física porque la librería que uso para acceder a los
ficheros, EFS \cite{website:efslib}, tenía un bug que la hacía entrar
en un bucle infinito al ejecutarse en una Nintendo DS. Así que he
tenido que desarrollarlo probando las cosas en el emulador DeSmuMe.

Cuando por fin han corregido la librería EFS, he podido probar el
juego en la videoconsola física y sólo he tenido que corregir dos
bugs.

\begin{itemize}
\item
Uno de ellos era que en el vector \texttt{faces\_} de la clase
\texttt{TextBoxHandler}, usaba el operador [] sin haber introducido
previamente nada en el vector.

\item
El otro fallo era un glitch de la pantalla principal, que al cambiar
de tarjeta, se imprimía por un momento el ``background'' donde está el
texto desplazado.\\
El fallo se producía al ajustar el desplazamiento del background en la
pantalla sin haber realmente cambiado. He podido arreglarlo fácilmente
añadiendo una nueva variable a la clase\\
\texttt{MainScreenHandler} que almacena el scroll vertical actual, y
sólo lo actualiza cuando se cambie de verdad.
\end{itemize}

