% Este archivo es parte de la memoria del proyecto fin de carrera
% de Manuel López Urbina. Protegida bajo la licencia GFDL.
% Para más información, la licencia completa viene incluida en el
% fichero fdl-1.3.tex

% Copyright (C) 2012 Manuel López Urbina

\section*{Agradecimientos}

Este proyecto significa la culminación de mi carrera, por lo que me gustaría dedicárselo a todas las personas que me han ayudado a conseguir acabarla.\\

En primer lugar me gustaría agradecerle a mi familia el apoyarme y ayudarme durante estos años, y el esfuerzo que han hecho para que yo haya podido llegar a este momento.\\

Agradecimientos a D. Arturo Morgado por su ayuda y dedicación durante la realización y dirección de este proyecto, así como su carácter amable y servicial que han hecho más ameno el trabajo realizado.\\

También me gustaría agradecérselo a mis compañeros, con los que tantos ratos inolvidables he pasado, y que tanto me han ayudado, entre ellos, agradecimientos a Noelia Sales Montes por la elaboración del logotipo del proyecto y a Andrés Francisco Aparicio por proporcionar un nombre al mismo. A Moisés Gautier Gómez por ser un excelente compañero tras numerosas horas de trabajo en el aula de robótica.\\

Por último quiero dedicarle este proyecto a todos los estudiantes de informática, en especial a todos los amantes del fascinante mundo de la robótica, a los que espero que mi trabajo les sea de utilidad.\\

\cleardoublepage

\section*{Licencia}

Este documento ha sido liberado bajo Licencia GFDL 1.3 (GNU Free Documentation License). Se incluyen los términos de la licencia en inglés al final del mismo.\\

Copyright (c) 2012 Manuel López Urbina.\\

Permission is granted to copy, distribute and/or modify this document under the terms of the GNU Free Documentation License, Version 1.3 or any later version published by the Free Software Foundation; with no Invariant Sections, no Front-Cover Texts, and no Back-Cover Texts. A copy of the license is included in the section entitled "GNU Free Documentation License".\\

\cleardoublepage

